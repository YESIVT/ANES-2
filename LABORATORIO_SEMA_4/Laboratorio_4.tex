% Options for packages loaded elsewhere
\PassOptionsToPackage{unicode}{hyperref}
\PassOptionsToPackage{hyphens}{url}
%
\documentclass[
]{article}
\usepackage{amsmath,amssymb}
\usepackage{lmodern}
\usepackage{iftex}
\ifPDFTeX
  \usepackage[T1]{fontenc}
  \usepackage[utf8]{inputenc}
  \usepackage{textcomp} % provide euro and other symbols
\else % if luatex or xetex
  \usepackage{unicode-math}
  \defaultfontfeatures{Scale=MatchLowercase}
  \defaultfontfeatures[\rmfamily]{Ligatures=TeX,Scale=1}
\fi
% Use upquote if available, for straight quotes in verbatim environments
\IfFileExists{upquote.sty}{\usepackage{upquote}}{}
\IfFileExists{microtype.sty}{% use microtype if available
  \usepackage[]{microtype}
  \UseMicrotypeSet[protrusion]{basicmath} % disable protrusion for tt fonts
}{}
\makeatletter
\@ifundefined{KOMAClassName}{% if non-KOMA class
  \IfFileExists{parskip.sty}{%
    \usepackage{parskip}
  }{% else
    \setlength{\parindent}{0pt}
    \setlength{\parskip}{6pt plus 2pt minus 1pt}}
}{% if KOMA class
  \KOMAoptions{parskip=half}}
\makeatother
\usepackage{xcolor}
\usepackage[margin=1in]{geometry}
\usepackage{color}
\usepackage{fancyvrb}
\newcommand{\VerbBar}{|}
\newcommand{\VERB}{\Verb[commandchars=\\\{\}]}
\DefineVerbatimEnvironment{Highlighting}{Verbatim}{commandchars=\\\{\}}
% Add ',fontsize=\small' for more characters per line
\usepackage{framed}
\definecolor{shadecolor}{RGB}{248,248,248}
\newenvironment{Shaded}{\begin{snugshade}}{\end{snugshade}}
\newcommand{\AlertTok}[1]{\textcolor[rgb]{0.94,0.16,0.16}{#1}}
\newcommand{\AnnotationTok}[1]{\textcolor[rgb]{0.56,0.35,0.01}{\textbf{\textit{#1}}}}
\newcommand{\AttributeTok}[1]{\textcolor[rgb]{0.77,0.63,0.00}{#1}}
\newcommand{\BaseNTok}[1]{\textcolor[rgb]{0.00,0.00,0.81}{#1}}
\newcommand{\BuiltInTok}[1]{#1}
\newcommand{\CharTok}[1]{\textcolor[rgb]{0.31,0.60,0.02}{#1}}
\newcommand{\CommentTok}[1]{\textcolor[rgb]{0.56,0.35,0.01}{\textit{#1}}}
\newcommand{\CommentVarTok}[1]{\textcolor[rgb]{0.56,0.35,0.01}{\textbf{\textit{#1}}}}
\newcommand{\ConstantTok}[1]{\textcolor[rgb]{0.00,0.00,0.00}{#1}}
\newcommand{\ControlFlowTok}[1]{\textcolor[rgb]{0.13,0.29,0.53}{\textbf{#1}}}
\newcommand{\DataTypeTok}[1]{\textcolor[rgb]{0.13,0.29,0.53}{#1}}
\newcommand{\DecValTok}[1]{\textcolor[rgb]{0.00,0.00,0.81}{#1}}
\newcommand{\DocumentationTok}[1]{\textcolor[rgb]{0.56,0.35,0.01}{\textbf{\textit{#1}}}}
\newcommand{\ErrorTok}[1]{\textcolor[rgb]{0.64,0.00,0.00}{\textbf{#1}}}
\newcommand{\ExtensionTok}[1]{#1}
\newcommand{\FloatTok}[1]{\textcolor[rgb]{0.00,0.00,0.81}{#1}}
\newcommand{\FunctionTok}[1]{\textcolor[rgb]{0.00,0.00,0.00}{#1}}
\newcommand{\ImportTok}[1]{#1}
\newcommand{\InformationTok}[1]{\textcolor[rgb]{0.56,0.35,0.01}{\textbf{\textit{#1}}}}
\newcommand{\KeywordTok}[1]{\textcolor[rgb]{0.13,0.29,0.53}{\textbf{#1}}}
\newcommand{\NormalTok}[1]{#1}
\newcommand{\OperatorTok}[1]{\textcolor[rgb]{0.81,0.36,0.00}{\textbf{#1}}}
\newcommand{\OtherTok}[1]{\textcolor[rgb]{0.56,0.35,0.01}{#1}}
\newcommand{\PreprocessorTok}[1]{\textcolor[rgb]{0.56,0.35,0.01}{\textit{#1}}}
\newcommand{\RegionMarkerTok}[1]{#1}
\newcommand{\SpecialCharTok}[1]{\textcolor[rgb]{0.00,0.00,0.00}{#1}}
\newcommand{\SpecialStringTok}[1]{\textcolor[rgb]{0.31,0.60,0.02}{#1}}
\newcommand{\StringTok}[1]{\textcolor[rgb]{0.31,0.60,0.02}{#1}}
\newcommand{\VariableTok}[1]{\textcolor[rgb]{0.00,0.00,0.00}{#1}}
\newcommand{\VerbatimStringTok}[1]{\textcolor[rgb]{0.31,0.60,0.02}{#1}}
\newcommand{\WarningTok}[1]{\textcolor[rgb]{0.56,0.35,0.01}{\textbf{\textit{#1}}}}
\usepackage{graphicx}
\makeatletter
\def\maxwidth{\ifdim\Gin@nat@width>\linewidth\linewidth\else\Gin@nat@width\fi}
\def\maxheight{\ifdim\Gin@nat@height>\textheight\textheight\else\Gin@nat@height\fi}
\makeatother
% Scale images if necessary, so that they will not overflow the page
% margins by default, and it is still possible to overwrite the defaults
% using explicit options in \includegraphics[width, height, ...]{}
\setkeys{Gin}{width=\maxwidth,height=\maxheight,keepaspectratio}
% Set default figure placement to htbp
\makeatletter
\def\fps@figure{htbp}
\makeatother
\setlength{\emergencystretch}{3em} % prevent overfull lines
\providecommand{\tightlist}{%
  \setlength{\itemsep}{0pt}\setlength{\parskip}{0pt}}
\setcounter{secnumdepth}{-\maxdimen} % remove section numbering
\ifLuaTeX
  \usepackage{selnolig}  % disable illegal ligatures
\fi
\IfFileExists{bookmark.sty}{\usepackage{bookmark}}{\usepackage{hyperref}}
\IfFileExists{xurl.sty}{\usepackage{xurl}}{} % add URL line breaks if available
\urlstyle{same} % disable monospaced font for URLs
\hypersetup{
  pdftitle={Laboratorio\_4.R},
  pdfauthor={yesiv},
  hidelinks,
  pdfcreator={LaTeX via pandoc}}

\title{Laboratorio\_4.R}
\author{yesiv}
\date{2023-02-28}

\begin{document}
\maketitle

\begin{Shaded}
\begin{Highlighting}[]
\CommentTok{\#Yesenia Villarreal Torres}
\CommentTok{\#1109559}
\CommentTok{\#Laboratorio 4}
\CommentTok{\#27 de febrero 2023}

\NormalTok{esp.url }\OtherTok{\textless{}{-}} \FunctionTok{paste0}\NormalTok{(}\StringTok{"https://raw.githubusercontent.com/mgtagle/"}\NormalTok{,}
                  \StringTok{"PrincipiosEstadistica2021/main/cuadro1.csv"}\NormalTok{)}
\NormalTok{inventario }\OtherTok{\textless{}{-}} \FunctionTok{read.csv}\NormalTok{(esp.url)}
\NormalTok{inventario}
\end{Highlighting}
\end{Shaded}

\begin{verbatim}
##    Arbol Fecha Especie Posicion Vecinos Diametros Altura
## 1      1    12       F        C       4      15.3  14.78
## 2      2    12       F        D       3      17.8  17.07
## 3      3     9       C        D       5      18.2  18.28
## 4      4     9       H        S       4       9.7   8.79
## 5      5     7       H        I       6      10.8  10.18
## 6      6    10       C        I       3      14.1  14.90
## 7      7    10       C        C       2      17.1  15.34
## 8      8    12       C        D       2      20.6  17.22
## 9      9    16       F        C       4      18.2  15.15
## 10    10    14       F        I       5      16.1  14.66
## 11    11     8       H        D       3      14.2  17.43
## 12    12     5       H        D       6      14.8  17.45
## 13    13    12       F        I       2      19.1  14.18
## 14    14     5       C        I       2      16.7  13.40
## 15    15    12       C        S       4      18.9  10.40
## 16    16    20       H        S       3      12.4  11.52
## 17    17    15       H        C       0      17.3  14.61
## 18    18    20       F        D       1      22.7  21.46
## 19    19    15       C        C       4      15.1  17.82
## 20    20    14       C        I       3      17.7  11.38
## 21    21    14       C        S       5      13.4   8.50
## 22    22    13       C        I       4      16.2  12.80
## 23    23    14       F        D       1      18.5  18.71
## 24    24    20       F        I       4      15.0  14.48
## 25    25    21       F        C       2      18.8  14.81
## 26    26     5       H        I       4      15.8  12.01
## 27    27     2       H        I       3      16.1  11.70
## 28    28    22       C        C       3      15.4  16.03
## 29    29    22       C        I       0      17.8  14.46
## 30    30    18       C        S       1      18.5   8.47
## 31    31    16       C        I       3      14.1  11.22
## 32    32    16       C        C       5      14.8  12.34
## 33    33    17       F        C       4      15.5  16.79
## 34    34    17       F        I       6      13.8  16.06
## 35    35    18       F        S       4      13.0  13.20
## 36    36    20       H        C       2      18.2  14.30
## 37    37    22       H        C       0      22.3  16.84
## 38    38    20       H        I       3      17.8  13.84
## 39    39    17       C        I       4      13.1  11.31
## 40    40    17       C        I       6      12.8  13.20
## 41    41    16       C        C       3      13.3  13.75
## 42    42    23       F        C       3      15.6  14.60
## 43    43    23       H        C       4      16.6  12.56
## 44    43    22       C        I       5      13.0  10.88
## 45    45    24       C        I       4      10.2  13.93
## 46    46    23       F        I       3      14.4  12.68
## 47    47    24       C        S       6       7.7  10.00
## 48    48    25       C        S       5       9.9   8.69
## 49    49    25       H        D       1      20.4  16.73
## 50    50    24       H        D       3      20.9  16.25
\end{verbatim}

\begin{Shaded}
\begin{Highlighting}[]
\FunctionTok{str}\NormalTok{(inventario)}
\end{Highlighting}
\end{Shaded}

\begin{verbatim}
## 'data.frame':    50 obs. of  7 variables:
##  $ Arbol    : int  1 2 3 4 5 6 7 8 9 10 ...
##  $ Fecha    : int  12 12 9 9 7 10 10 12 16 14 ...
##  $ Especie  : chr  "F" "F" "C" "H" ...
##  $ Posicion : chr  "C" "D" "D" "S" ...
##  $ Vecinos  : int  4 3 5 4 6 3 2 2 4 5 ...
##  $ Diametros: num  15.3 17.8 18.2 9.7 10.8 14.1 17.1 20.6 18.2 16.1 ...
##  $ Altura   : num  14.78 17.07 18.28 8.79 10.18 ...
\end{verbatim}

\begin{Shaded}
\begin{Highlighting}[]
\FunctionTok{dim}\NormalTok{(inventario)}
\end{Highlighting}
\end{Shaded}

\begin{verbatim}
## [1] 50  7
\end{verbatim}

\begin{Shaded}
\begin{Highlighting}[]
\FunctionTok{head}\NormalTok{(inventario, }\AttributeTok{n =} \DecValTok{5}\NormalTok{)}
\end{Highlighting}
\end{Shaded}

\begin{verbatim}
##   Arbol Fecha Especie Posicion Vecinos Diametros Altura
## 1     1    12       F        C       4      15.3  14.78
## 2     2    12       F        D       3      17.8  17.07
## 3     3     9       C        D       5      18.2  18.28
## 4     4     9       H        S       4       9.7   8.79
## 5     5     7       H        I       6      10.8  10.18
\end{verbatim}

\begin{Shaded}
\begin{Highlighting}[]
\FunctionTok{tail}\NormalTok{(inventario, }\AttributeTok{n =} \DecValTok{5}\NormalTok{)}
\end{Highlighting}
\end{Shaded}

\begin{verbatim}
##    Arbol Fecha Especie Posicion Vecinos Diametros Altura
## 46    46    23       F        I       3      14.4  12.68
## 47    47    24       C        S       6       7.7  10.00
## 48    48    25       C        S       5       9.9   8.69
## 49    49    25       H        D       1      20.4  16.73
## 50    50    24       H        D       3      20.9  16.25
\end{verbatim}

\begin{Shaded}
\begin{Highlighting}[]
\FunctionTok{names}\NormalTok{(inventario)}
\end{Highlighting}
\end{Shaded}

\begin{verbatim}
## [1] "Arbol"     "Fecha"     "Especie"   "Posicion"  "Vecinos"   "Diametros"
## [7] "Altura"
\end{verbatim}

\begin{Shaded}
\begin{Highlighting}[]
\FunctionTok{colnames}\NormalTok{(inventario)}
\end{Highlighting}
\end{Shaded}

\begin{verbatim}
## [1] "Arbol"     "Fecha"     "Especie"   "Posicion"  "Vecinos"   "Diametros"
## [7] "Altura"
\end{verbatim}

\begin{Shaded}
\begin{Highlighting}[]
\FunctionTok{summary}\NormalTok{(inventario)}
\end{Highlighting}
\end{Shaded}

\begin{verbatim}
##      Arbol           Fecha         Especie            Posicion        
##  Min.   : 1.00   Min.   : 2.00   Length:50          Length:50         
##  1st Qu.:13.25   1st Qu.:12.00   Class :character   Class :character  
##  Median :25.50   Median :16.00   Mode  :character   Mode  :character  
##  Mean   :25.48   Mean   :15.94                                        
##  3rd Qu.:37.75   3rd Qu.:20.75                                        
##  Max.   :50.00   Max.   :25.00                                        
##     Vecinos       Diametros         Altura     
##  Min.   :0.00   Min.   : 7.70   Min.   : 8.47  
##  1st Qu.:2.25   1st Qu.:13.88   1st Qu.:11.78  
##  Median :3.00   Median :15.70   Median :14.24  
##  Mean   :3.34   Mean   :15.79   Mean   :13.94  
##  3rd Qu.:4.00   3rd Qu.:18.10   3rd Qu.:16.05  
##  Max.   :6.00   Max.   :22.70   Max.   :21.46
\end{verbatim}

\begin{Shaded}
\begin{Highlighting}[]
\CommentTok{\#dimensiones (num filas y columnas)}

\FunctionTok{dim}\NormalTok{(inventario)}
\end{Highlighting}
\end{Shaded}

\begin{verbatim}
## [1] 50  7
\end{verbatim}

\begin{Shaded}
\begin{Highlighting}[]
\CommentTok{\#nombre de las primeras 5 columnas}

\FunctionTok{names}\NormalTok{(inventario[ ,}\DecValTok{1}\SpecialCharTok{:}\DecValTok{5}\NormalTok{])}
\end{Highlighting}
\end{Shaded}

\begin{verbatim}
## [1] "Arbol"    "Fecha"    "Especie"  "Posicion" "Vecinos"
\end{verbatim}

\begin{Shaded}
\begin{Highlighting}[]
\FunctionTok{summary}\NormalTok{(inventario[ ,}\DecValTok{3}\SpecialCharTok{:}\DecValTok{5}\NormalTok{])}
\end{Highlighting}
\end{Shaded}

\begin{verbatim}
##    Especie            Posicion            Vecinos    
##  Length:50          Length:50          Min.   :0.00  
##  Class :character   Class :character   1st Qu.:2.25  
##  Mode  :character   Mode  :character   Median :3.00  
##                                        Mean   :3.34  
##                                        3rd Qu.:4.00  
##                                        Max.   :6.00
\end{verbatim}

\begin{Shaded}
\begin{Highlighting}[]
\FunctionTok{is.factor}\NormalTok{(inventario}\SpecialCharTok{$}\NormalTok{Posicion)}
\end{Highlighting}
\end{Shaded}

\begin{verbatim}
## [1] FALSE
\end{verbatim}

\begin{Shaded}
\begin{Highlighting}[]
\NormalTok{inventario}\SpecialCharTok{$}\NormalTok{Posicion }\OtherTok{\textless{}{-}} \FunctionTok{factor}\NormalTok{(inventario}\SpecialCharTok{$}\NormalTok{Posicion)}
\FunctionTok{is.factor}\NormalTok{(inventario}\SpecialCharTok{$}\NormalTok{Posicion)}
\end{Highlighting}
\end{Shaded}

\begin{verbatim}
## [1] TRUE
\end{verbatim}

\begin{Shaded}
\begin{Highlighting}[]
\FunctionTok{summary}\NormalTok{(inventario[ ,}\DecValTok{3}\SpecialCharTok{:}\DecValTok{5}\NormalTok{])}
\end{Highlighting}
\end{Shaded}

\begin{verbatim}
##    Especie          Posicion    Vecinos    
##  Length:50          C:14     Min.   :0.00  
##  Class :character   D: 9     1st Qu.:2.25  
##  Mode  :character   I:19     Median :3.00  
##                     S: 8     Mean   :3.34  
##                              3rd Qu.:4.00  
##                              Max.   :6.00
\end{verbatim}

\begin{Shaded}
\begin{Highlighting}[]
\FunctionTok{summary}\NormalTok{(inventario)}
\end{Highlighting}
\end{Shaded}

\begin{verbatim}
##      Arbol           Fecha         Especie          Posicion    Vecinos    
##  Min.   : 1.00   Min.   : 2.00   Length:50          C:14     Min.   :0.00  
##  1st Qu.:13.25   1st Qu.:12.00   Class :character   D: 9     1st Qu.:2.25  
##  Median :25.50   Median :16.00   Mode  :character   I:19     Median :3.00  
##  Mean   :25.48   Mean   :15.94                      S: 8     Mean   :3.34  
##  3rd Qu.:37.75   3rd Qu.:20.75                               3rd Qu.:4.00  
##  Max.   :50.00   Max.   :25.00                               Max.   :6.00  
##    Diametros         Altura     
##  Min.   : 7.70   Min.   : 8.47  
##  1st Qu.:13.88   1st Qu.:11.78  
##  Median :15.70   Median :14.24  
##  Mean   :15.79   Mean   :13.94  
##  3rd Qu.:18.10   3rd Qu.:16.05  
##  Max.   :22.70   Max.   :21.46
\end{verbatim}

\begin{Shaded}
\begin{Highlighting}[]
\CommentTok{\# Tablas de frecuencia {-}{-}{-}{-}{-}{-}{-}{-}{-}{-}{-}{-}{-}{-}{-}{-}{-}{-}{-}{-}{-}{-}{-}{-}{-}{-}{-}{-}{-}{-}{-}{-}{-}{-}{-}{-}{-}{-}{-}{-}{-}{-}{-}{-}{-}{-}{-}{-}{-}{-}{-}{-}}


\NormalTok{freq\_position }\OtherTok{\textless{}{-}} \FunctionTok{table}\NormalTok{(inventario}\SpecialCharTok{$}\NormalTok{Posicion)}
\NormalTok{freq\_position}
\end{Highlighting}
\end{Shaded}

\begin{verbatim}
## 
##  C  D  I  S 
## 14  9 19  8
\end{verbatim}

\begin{Shaded}
\begin{Highlighting}[]
\NormalTok{prop\_position }\OtherTok{\textless{}{-}}\NormalTok{ freq\_position }\SpecialCharTok{/} \FunctionTok{sum}\NormalTok{(freq\_position)}
\NormalTok{prop\_position}
\end{Highlighting}
\end{Shaded}

\begin{verbatim}
## 
##    C    D    I    S 
## 0.28 0.18 0.38 0.16
\end{verbatim}

\begin{Shaded}
\begin{Highlighting}[]
\CommentTok{\#Si desea expresar las proporciones como porcentajes, multiplique prop\_position por 100:}
\NormalTok{perc\_position }\OtherTok{=} \DecValTok{100} \SpecialCharTok{*}\NormalTok{ prop\_position}
\NormalTok{perc\_position}
\end{Highlighting}
\end{Shaded}

\begin{verbatim}
## 
##  C  D  I  S 
## 28 18 38 16
\end{verbatim}

\begin{Shaded}
\begin{Highlighting}[]
\CommentTok{\# Gráficas barplot y pie {-}{-}{-}{-}{-}{-}{-}{-}{-}{-}{-}{-}{-}{-}{-}{-}{-}{-}{-}{-}{-}{-}{-}{-}{-}{-}{-}{-}{-}{-}{-}{-}{-}{-}{-}{-}{-}{-}{-}{-}{-}{-}{-}{-}{-}{-}{-}{-}{-}{-}}

\CommentTok{\#Gráficas de barras (barplot)}
\CommentTok{\#Gráficas de pastel (pie)}

\FunctionTok{barplot}\NormalTok{(freq\_position, }\AttributeTok{las =} \DecValTok{1}\NormalTok{, }\AttributeTok{border =} \ConstantTok{NA}\NormalTok{, }\AttributeTok{cex.names =} \FloatTok{0.7}\NormalTok{)}
\end{Highlighting}
\end{Shaded}

\includegraphics{Laboratorio_4_files/figure-latex/unnamed-chunk-1-1.pdf}

\begin{Shaded}
\begin{Highlighting}[]
\CommentTok{\#las = 1: muestra las frecuencias perpendiculares al eje{-}y.}
\CommentTok{\#border = NA: elimina el borde negro alrededor de las barras.}
\CommentTok{\#cex.names = 0.7: reduce los tamaños de las etiquetas de categoría (para que todas quepan en el gráfico).}



\CommentTok{\#Gráfico circular o pie. El otro tipo común de gráfico para ver frecuencias es un gráfico circular. R proporciona la función pie() para producir estos gráficos:}
\FunctionTok{pie}\NormalTok{(freq\_position, }\AttributeTok{col =} \FunctionTok{topo.colors}\NormalTok{(}\DecValTok{4}\NormalTok{),}
\AttributeTok{labels =} \FunctionTok{paste}\NormalTok{(}\FunctionTok{levels}\NormalTok{(inventario}\SpecialCharTok{$}\NormalTok{Posicion), }\FunctionTok{round}\NormalTok{(perc\_position, }\DecValTok{2}\NormalTok{), }\StringTok{" \%"}\NormalTok{))}
\end{Highlighting}
\end{Shaded}

\includegraphics{Laboratorio_4_files/figure-latex/unnamed-chunk-1-2.pdf}

\begin{Shaded}
\begin{Highlighting}[]
  \CommentTok{\#Autoestudio}

  \CommentTok{\# topo.colors es una paleta de colores pre establecidas en R y}
  \CommentTok{\# el paréntesis indica el \# de colores a usar  }
\CommentTok{\#Completar una tabla de frecuencia y su representación gráfica (barplot y pie) para la variable Especie del conjunto de datos inventario}
\NormalTok{freq\_Especie }\OtherTok{\textless{}{-}} \FunctionTok{table}\NormalTok{(inventario}\SpecialCharTok{$}\NormalTok{Especie)}
\NormalTok{freq\_Especie}
\end{Highlighting}
\end{Shaded}

\begin{verbatim}
## 
##  C  F  H 
## 22 14 14
\end{verbatim}

\begin{Shaded}
\begin{Highlighting}[]
\NormalTok{prop\_Especie }\OtherTok{\textless{}{-}}\NormalTok{ freq\_Especie }\SpecialCharTok{/} \FunctionTok{sum}\NormalTok{(freq\_Especie)}
\NormalTok{prop\_Especie}
\end{Highlighting}
\end{Shaded}

\begin{verbatim}
## 
##    C    F    H 
## 0.44 0.28 0.28
\end{verbatim}

\begin{Shaded}
\begin{Highlighting}[]
\NormalTok{perc\_Especie }\OtherTok{=} \DecValTok{100} \SpecialCharTok{*}\NormalTok{ prop\_Especie}
\NormalTok{perc\_Especie}
\end{Highlighting}
\end{Shaded}

\begin{verbatim}
## 
##  C  F  H 
## 44 28 28
\end{verbatim}

\begin{Shaded}
\begin{Highlighting}[]
\FunctionTok{barplot}\NormalTok{(freq\_Especie, }\AttributeTok{las =} \DecValTok{1}\NormalTok{, }\AttributeTok{border =} \ConstantTok{NA}\NormalTok{, }\AttributeTok{cex.names =} \FloatTok{0.7}\NormalTok{)}
\end{Highlighting}
\end{Shaded}

\includegraphics{Laboratorio_4_files/figure-latex/unnamed-chunk-1-3.pdf}

\begin{Shaded}
\begin{Highlighting}[]
\FunctionTok{pie}\NormalTok{(freq\_Especie, }\AttributeTok{col =} \FunctionTok{topo.colors}\NormalTok{(}\DecValTok{3}\NormalTok{),}
    
    \AttributeTok{labels =} \FunctionTok{paste}\NormalTok{(}\FunctionTok{levels}\NormalTok{(inventario}\SpecialCharTok{$}\NormalTok{Especie), }\FunctionTok{round}\NormalTok{(perc\_Especie, }\DecValTok{2}\NormalTok{), }\StringTok{" \%"}\NormalTok{))}
\end{Highlighting}
\end{Shaded}

\includegraphics{Laboratorio_4_files/figure-latex/unnamed-chunk-1-4.pdf}

\begin{Shaded}
\begin{Highlighting}[]
\CommentTok{\# Representación de variables cuantitativas {-}{-}{-}{-}{-}{-}{-}{-}{-}{-}{-}{-}{-}{-}{-}{-}{-}{-}{-}{-}{-}{-}{-}{-}{-}{-}{-}{-}{-}{-}{-}}

\CommentTok{\#histogramas}
\CommentTok{\#boxplots o gráfica de cajas}

\CommentTok{\#Histogramas}
\CommentTok{\#Un histograma es un tipo de gráfico que muestra la distribución de datos numéricos}

\NormalTok{diam\_hist }\OtherTok{\textless{}{-}} \FunctionTok{hist}\NormalTok{(inventario}\SpecialCharTok{$}\NormalTok{Diametros, }\AttributeTok{las =} \DecValTok{1}\NormalTok{, }\AttributeTok{col =} \StringTok{\textquotesingle{}yellow\textquotesingle{}}\NormalTok{)}
\end{Highlighting}
\end{Shaded}

\includegraphics{Laboratorio_4_files/figure-latex/unnamed-chunk-1-5.pdf}

\begin{Shaded}
\begin{Highlighting}[]
\NormalTok{diam\_hist}
\end{Highlighting}
\end{Shaded}

\begin{verbatim}
## $breaks
##  [1]  6  8 10 12 14 16 18 20 22 24
## 
## $counts
## [1]  1  2  2  8 13 11  8  3  2
## 
## $density
## [1] 0.01 0.02 0.02 0.08 0.13 0.11 0.08 0.03 0.02
## 
## $mids
## [1]  7  9 11 13 15 17 19 21 23
## 
## $xname
## [1] "inventario$Diametros"
## 
## $equidist
## [1] TRUE
## 
## attr(,"class")
## [1] "histogram"
\end{verbatim}

\begin{Shaded}
\begin{Highlighting}[]
\CommentTok{\#breaks: puntos de ruptura (corte) de los intervalos de clase}
\CommentTok{\#counts: número de observaciones en cada categoría}
\CommentTok{\#density: densidad}
\CommentTok{\#mids: punto central del intervalo}
\CommentTok{\#xname: nombre del objeto (variable) que se esta graficando}
\CommentTok{\#equidist: ¿Los categorías tienen el mismo ancho?}
\CommentTok{\# attr: Tipo de clase}

\NormalTok{diam\_hist}\SpecialCharTok{$}\NormalTok{breaks}
\end{Highlighting}
\end{Shaded}

\begin{verbatim}
##  [1]  6  8 10 12 14 16 18 20 22 24
\end{verbatim}

\begin{Shaded}
\begin{Highlighting}[]
\NormalTok{diam\_hist}\SpecialCharTok{$}\NormalTok{mids}
\end{Highlighting}
\end{Shaded}

\begin{verbatim}
## [1]  7  9 11 13 15 17 19 21 23
\end{verbatim}

\begin{Shaded}
\begin{Highlighting}[]
\NormalTok{diam\_hist}\SpecialCharTok{$}\NormalTok{counts}
\end{Highlighting}
\end{Shaded}

\begin{verbatim}
## [1]  1  2  2  8 13 11  8  3  2
\end{verbatim}

\begin{Shaded}
\begin{Highlighting}[]
\NormalTok{diam\_hist}\SpecialCharTok{$}\NormalTok{density}
\end{Highlighting}
\end{Shaded}

\begin{verbatim}
## [1] 0.01 0.02 0.02 0.08 0.13 0.11 0.08 0.03 0.02
\end{verbatim}

\begin{Shaded}
\begin{Highlighting}[]
\NormalTok{diam\_hist}\SpecialCharTok{$}\NormalTok{xname}
\end{Highlighting}
\end{Shaded}

\begin{verbatim}
## [1] "inventario$Diametros"
\end{verbatim}

\begin{Shaded}
\begin{Highlighting}[]
\NormalTok{diam\_hist}\SpecialCharTok{$}\NormalTok{equidist}
\end{Highlighting}
\end{Shaded}

\begin{verbatim}
## [1] TRUE
\end{verbatim}

\begin{Shaded}
\begin{Highlighting}[]
\NormalTok{diam\_hist}\SpecialCharTok{$}\NormalTok{attr}
\end{Highlighting}
\end{Shaded}

\begin{verbatim}
## NULL
\end{verbatim}

\begin{Shaded}
\begin{Highlighting}[]
\NormalTok{h1 }\OtherTok{\textless{}{-}} \FunctionTok{hist}\NormalTok{(inventario}\SpecialCharTok{$}\NormalTok{Diametros, }\AttributeTok{xaxt =} \StringTok{"n"}\NormalTok{,}
           \AttributeTok{breaks =} \FunctionTok{c}\NormalTok{(}\DecValTok{6}\NormalTok{, }\DecValTok{8}\NormalTok{, }\DecValTok{10}\NormalTok{, }\DecValTok{12}\NormalTok{, }\DecValTok{14}\NormalTok{, }\DecValTok{16}\NormalTok{, }\DecValTok{18}\NormalTok{, }\DecValTok{20}\NormalTok{, }\DecValTok{22}\NormalTok{,}\DecValTok{24}\NormalTok{),}
           \AttributeTok{col =} \StringTok{"\#00cFDFDF"}\NormalTok{, }\AttributeTok{xlab=}\StringTok{"Diámetros (cm)"}\NormalTok{,}
           \AttributeTok{ylab=} \StringTok{"Frecuencias"}\NormalTok{,}
           \AttributeTok{main =} \StringTok{""}\NormalTok{,}
           \AttributeTok{las =} \DecValTok{1}\NormalTok{,}
           \AttributeTok{ylim =} \FunctionTok{c}\NormalTok{(}\DecValTok{0}\NormalTok{,}\DecValTok{14}\NormalTok{))}
\FunctionTok{axis}\NormalTok{(}\DecValTok{1}\NormalTok{, h1}\SpecialCharTok{$}\NormalTok{mids)}
\end{Highlighting}
\end{Shaded}

\includegraphics{Laboratorio_4_files/figure-latex/unnamed-chunk-1-6.pdf}

\begin{Shaded}
\begin{Highlighting}[]
\CommentTok{\# Autoestudio Realizar el mismo procedimiento para la variable Alt {-}{-}{-}{-}{-}{-}{-}{-}}

\NormalTok{altura\_hist }\OtherTok{\textless{}{-}} \FunctionTok{hist}\NormalTok{(inventario}\SpecialCharTok{$}\NormalTok{Altura, }\AttributeTok{las =} \DecValTok{1}\NormalTok{, }\AttributeTok{col =} \StringTok{\textquotesingle{}yellow\textquotesingle{}}\NormalTok{)}
\end{Highlighting}
\end{Shaded}

\includegraphics{Laboratorio_4_files/figure-latex/unnamed-chunk-1-7.pdf}

\begin{Shaded}
\begin{Highlighting}[]
\NormalTok{altura\_hist}
\end{Highlighting}
\end{Shaded}

\begin{verbatim}
## $breaks
## [1]  8 10 12 14 16 18 20 22
## 
## $counts
## [1]  5  8 11 12 11  2  1
## 
## $density
## [1] 0.05 0.08 0.11 0.12 0.11 0.02 0.01
## 
## $mids
## [1]  9 11 13 15 17 19 21
## 
## $xname
## [1] "inventario$Altura"
## 
## $equidist
## [1] TRUE
## 
## attr(,"class")
## [1] "histogram"
\end{verbatim}

\begin{Shaded}
\begin{Highlighting}[]
\NormalTok{altura\_hist}\SpecialCharTok{$}\NormalTok{breaks}
\end{Highlighting}
\end{Shaded}

\begin{verbatim}
## [1]  8 10 12 14 16 18 20 22
\end{verbatim}

\begin{Shaded}
\begin{Highlighting}[]
\NormalTok{altura\_hist}\SpecialCharTok{$}\NormalTok{mids}
\end{Highlighting}
\end{Shaded}

\begin{verbatim}
## [1]  9 11 13 15 17 19 21
\end{verbatim}

\begin{Shaded}
\begin{Highlighting}[]
\NormalTok{altura\_hist}\SpecialCharTok{$}\NormalTok{counts}
\end{Highlighting}
\end{Shaded}

\begin{verbatim}
## [1]  5  8 11 12 11  2  1
\end{verbatim}

\begin{Shaded}
\begin{Highlighting}[]
\NormalTok{altura\_hist}\SpecialCharTok{$}\NormalTok{density}
\end{Highlighting}
\end{Shaded}

\begin{verbatim}
## [1] 0.05 0.08 0.11 0.12 0.11 0.02 0.01
\end{verbatim}

\begin{Shaded}
\begin{Highlighting}[]
\NormalTok{altura\_hist}\SpecialCharTok{$}\NormalTok{xname}
\end{Highlighting}
\end{Shaded}

\begin{verbatim}
## [1] "inventario$Altura"
\end{verbatim}

\begin{Shaded}
\begin{Highlighting}[]
\NormalTok{altura\_hist}\SpecialCharTok{$}\NormalTok{equidist}
\end{Highlighting}
\end{Shaded}

\begin{verbatim}
## [1] TRUE
\end{verbatim}

\begin{Shaded}
\begin{Highlighting}[]
\NormalTok{altura\_hist}\SpecialCharTok{$}\NormalTok{attr}
\end{Highlighting}
\end{Shaded}

\begin{verbatim}
## NULL
\end{verbatim}

\begin{Shaded}
\begin{Highlighting}[]
\NormalTok{h2 }\OtherTok{\textless{}{-}} \FunctionTok{hist}\NormalTok{(inventario}\SpecialCharTok{$}\NormalTok{Altura, }\AttributeTok{xaxt =} \StringTok{"n"}\NormalTok{,}
           \AttributeTok{breaks =} \FunctionTok{c}\NormalTok{(}\DecValTok{8}\NormalTok{, }\DecValTok{10}\NormalTok{, }\DecValTok{12}\NormalTok{, }\DecValTok{14}\NormalTok{, }\DecValTok{16}\NormalTok{, }\DecValTok{18}\NormalTok{, }\DecValTok{20}\NormalTok{, }\DecValTok{22}\NormalTok{),}
           \AttributeTok{col =} \StringTok{"\#12444444"}\NormalTok{, }\AttributeTok{xlab=}\StringTok{"Métros (m)"}\NormalTok{,}
           \AttributeTok{ylab=} \StringTok{"Frecuencias"}\NormalTok{,}
           \AttributeTok{main =} \StringTok{""}\NormalTok{,}
           \AttributeTok{las =} \DecValTok{1}\NormalTok{,}
           \AttributeTok{ylim =} \FunctionTok{c}\NormalTok{(}\DecValTok{0}\NormalTok{,}\DecValTok{14}\NormalTok{))}
\FunctionTok{axis}\NormalTok{(}\DecValTok{1}\NormalTok{, h1}\SpecialCharTok{$}\NormalTok{mids)}
\end{Highlighting}
\end{Shaded}

\includegraphics{Laboratorio_4_files/figure-latex/unnamed-chunk-1-8.pdf}

\begin{Shaded}
\begin{Highlighting}[]
\CommentTok{\#FIN 1}
\end{Highlighting}
\end{Shaded}


\end{document}
